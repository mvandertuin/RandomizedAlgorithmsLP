\documentclass{acm_proc_article-sp}
\usepackage{amsmath}
\usepackage{amsfonts}
\usepackage{amssymb}
\usepackage[algoruled]{algorithm2e}

\begin{document}

\title{Randomized Linear Programming}
\subtitle{Lab Report Randomized Algorithms IN4337}
\numberofauthors{2} 
\author{
\alignauthor
Koos van der Linden \\ \affaddr{4133145}
% 2nd. author
\alignauthor
Marieke van der Tuin \\ \affaddr{4079299}
}
\maketitle
\begin{abstract}
Abstract here...
\end{abstract}

\section{Problem introduction}
The Linear programming problem is concerned with finding the optimum of an objective function limited to a set of constraints. The standard form of a linear program is written in equation \ref{eq:standard}, with $x$ the vector variables to be determined and $A$ the matrix of constraints that should be less or equal to vector $b$.
\begin{equation}
\label{eq:standard}
\begin{split}
\text{minimize } & c^t x \\
\text{subject to } &  Ax\leq b \\
\text{and } & x \geq 0
\end{split}
\end{equation}

\section{Description of used methods}
\subsection{Simplex}
The Simplex algorithm \cite{dantzig1951maximization} is the most basic algorithm for solving linear programming problems. This algorithm uses the standard form as an input and has a time complexity of polynomial time on average, but has a worst-case complexity of exponential time. 

\subsection{Gurobi Optimizer}
The Gurobi optimizer is a commercial solver for linear programming problems. It gives several interfaces to use the solver, including Java. Since it belongs to one of the best LP-solvers, it gives the ability of solving self-generated linear programming problems very quickly for checking for feasibility.

\subsection{SampLP}
The SampLP algorithm (\ref{alg:samplp}) was introduced by Clarkson\cite{clarkson1988vegas}. This randomized Las Vegas algorithm uses random sampling to throw away redundant constraints quickly. It only gives a speed-up for instances with a small dimension: if $n<9d^2$, it calls the Simplex algorithm. 

%TODO: add computational complexity

\begin{algorithm}[h]
\label{alg:samplp}
\caption{SampLP}
\KwIn{A set of constraints $H$}
\KwOut{The optimum $\beta(H)$}
$S \gets \phi$\;
\If{$n<9d^2$}{
	\Return{Simplex($H$)}\;
}
\Else{
	$V \gets H$ \;
	$S \gets \phi$ \;
	\While{$|V| > 0$}{
		Choose $R\subset H\setminus S$ at random, with $|R| = r = min{d*\sqrt{n},|H\setminus S|}$ \;
		$x \gets SampLP(R\cup S)$ \;
		$V \gets {h \in H|vertex defined by x violates h}$\;
		\If{$|V|\leq 2*\sqrt{n}$}{
			$S \gets S \cup V$ \;
		}		
	}
	\Return{$x$}\;
}
\end{algorithm}

\subsection{IterSampLP}
The IterSampLP algorithm (\ref{alg:itersamplp}) is an improvement of the SampLP algorithm and was also introduced by Clarkson\cite{clarkson1995vegas}. It uses weighted constraints. For each iterative step it reweights the more interesting constraints to increase the probability of including such a constraint in the random sample.

%TODO: add computational complexity

\begin{algorithm}[h]
\label{alg:itersamplp}
\caption{IterSampLP}
\KwIn{A set of constraints $H$}
\KwOut{The optimum $\beta(H)$}
$\forall h \in H, w_h \gets 1$ \;
\If{$n<9d^2$}{
	\Return{Simplex($H$)}\;
}
\Else{
	$V \gets H$ \;
	\While{$|V| > 0$}{
		Choose $R\subset H$ at random, with $|R| = r = 9*d^2$ \;
		$x \gets Simplex(R)$ \;
		$V \gets {h \in H|vertex defined by x violates h}$\;
		\If{$\displaystyle\sum_{h\in V} w_h \leq (2 * \displaystyle\sum_{h \in H} w_h)/(9*d-1)$} {
			$\forall h \in V, w_h \gets 2*w_h$ \;
		}		
	}
	\Return{$x$}\;
}
\end{algorithm}

\section{Experimental results}
%benchmark miplib: which instances to choose?

%runtime [ms] op benchmark instances (x-as=#of constraints en x-as=#of vars)
%aantal iteraties binnen simplexApache voor sampleLP & iterSampLP

\section{Analysis of results}
Compare between methods: runtime and asymptotic bounds.

\section{Discussion of the benefit of randomization for the problem}

\section{Conclusions}


\bibliographystyle{abbrv}
\bibliography{sigproc}  % sigproc.bib is the name of the Bibliography in this case
\balancecolumns
\end{document}
