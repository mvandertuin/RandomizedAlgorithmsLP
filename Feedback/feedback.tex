\documentclass[nocopyrightspace]{acm_proc_article-sp}
\usepackage{amsmath}
\usepackage{amsfonts}
\usepackage{amssymb}
\usepackage[algoruled]{algorithm2e}
\usepackage{tabularx}

%Remove the box from the bottom left from the template
\makeatletter
\def\@copyrightspace{\relax}
\makeatother

\begin{document}

\title{Choice Coordination and Byzantine Agreement}
\subtitle{Feedback Report Randomized Algorithms IN4337}
\numberofauthors{2} 
\author{
\alignauthor
Koos van der Linden \\ \affaddr{4133145}
% 2nd. author
\alignauthor
Marieke van der Tuin \\ \affaddr{4079299}
}
\maketitle
\section{Structure}
We have several remarks on the structure of the report. The report is readable on its own, which is good. The problem is introduced and variables are explained. When the algorithms are introduced though, this is a somewhat unclear. The difference between $C_i$ and $R_i$ is not explained.\\
The report begins with an introduction that has little use. An introduction should contain at least a motivation. Section 2 should explain the problem and the model, which it does properly. But this section also contains other information such as the running time of the algorithms that are yet to be introduced. Furthermore it mentions the running time of a deterministic algorithm for the choice coordination problem (CPP): such facts should be in the introduction. This is true for a large part of section 2. \\ 
The included pseudo-code in section three is very useful. It is sufficient to understand the algorithm. A little more explanation of the use of variables could be useful. \\
Before the experimental results are shown, an explanation of the experimental method should be included. What is measured, and why is this measured? The plots that are included are clear. The box-plot in figure 1 could have been left out. \\
We would suggest that section 5 should follow section 3. It is better to start with the theoretical analysis and compare this with the experimental results. Are the results as expected?\\
In section 2 it is mentioned that  a deterministic algorithm for CPP exists with running time $\Omega(n^\frac{1}{3})$. For the Byzantine Agreement (BA) a deterministic algorithm exists that requires $t+1$ round. No further reference is made to this. This is not compared with the running time of the randomized algorithm. The discussion of the benefit of randomization is missing in the report. \\
We also miss the comparison of the results with the theoretical analysis in section 5 and the constant running time and success probability mentioned in section 2. No conclusions are drawn from the experimental results, the results are not even discussed. \\
Section 5, the analysis, is done neatly. This section gives an answer to Exercise 12.20 and 12.21 and to problem 12.27

\section{Choice Coordination Problem}
For CPP we have several concerns. First of all, the algorithm for ASYNCH-CPP is given, but it is not analysed. It is said that the analysis for SYNCH-CCP also holds for ASYNCH-CCP. But ASYNCH-CPP is not tested, nor discussed, nor shown that it is equal to SYNCH-CPP, so we are not convinced. Is it even implemented? \\
In section 2 the probability of success of CPP is mentioned, but this is not explained. In the results this success probability is not tested. And what conclusions can be drawn from the fact that the number of iterations closely resemble a geometric distribution with $p=\frac{1}{2}$? \\


Complexity methods are given but not proven.



CPP is not discussed theoretically. 

No plot is included with n on the horizontal axe to convince that the running time is constant for CCP. Why $n=m=2$? This is not explained

\section{Byzantine Agreement}
Complexity methods are given but not proven.
Choice of $L$, $H$, and $G$ for ByzGen is not explained, nor are these variables explained

Why is t=4/n chosen, with t<n/8 as requirement. n<6 is derived in the analysis. Why is no conclusions drawn from this.
Why do the results show successful instances for t<n/8<n/6<n/4?

Algorithm 3 does not stop. This is explained later, so refer to it.

Figure 5: these are not averages. Why not?

The number of messages is max $O(n^2)$

References to original publisher of byzgen is missing (only referred to the book).
In this reference it is shown that t<n/4 is the bound. This is a better bound than t=n/6. The conclusion that no better bound can be found is thus incorrect.

Results are not checked for validity. Probability of success is not discussed, only for $t=n/4$


\end{document}
