\documentclass[nocopyrightspace]{acm_proc_article-sp}
\usepackage{amsmath}
\usepackage{amsfonts}
\usepackage{amssymb}
\usepackage[algoruled]{algorithm2e}
\usepackage{tabularx}

%Remove the box from the bottom left from the template
\makeatletter
\def\@copyrightspace{\relax}
\makeatother

\begin{document}

\title{Choice Coordination and Byzantine Agreement}
\subtitle{Feedback Report Randomized Algorithms IN4337}
\numberofauthors{2} 
\author{
\alignauthor
Koos van der Linden \\ \affaddr{4133145}
% 2nd. author
\alignauthor
Marieke van der Tuin \\ \affaddr{4079299}
}
\maketitle

Readable on its own

Problem introduction and motivation (motivation missing)
Introduction too short

Pseudocode is useful
Complexity methods are given. 

Asynch-CPP is not tested, nor discussed, nor shown that it is equal to Synch-CPP, so we are not convinced.

Experimental Plots are included.

Results are not compared with asymtoptic bounds, and not with each other.
No plot is included with n on the horizontal axe to convince that the running time is constant

Theoretical analysis before the experimental results

Why is t=4/n chosen, with t<n/8 as requirement. n<6 is derived in the analysis. Why is no conclusions drawn from this

Figure 5: these are not averages. Why not?

Discussion of the benefit of randomization for the problem (missing)

What is the connection with the analysis and the results

References to original publisher of byzgen is missing (only referred to the book).
In this reference it is shown that t<n/4 is the bound. This is a better bound than t=n/6. The conclusion that no better bound can be found is thus incorrect.

The number of messages is max O(n^2)


\end{document}
