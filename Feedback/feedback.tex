\documentclass[nocopyrightspace]{acm_proc_article-sp}
\usepackage{amsmath}
\usepackage{amsfonts}
\usepackage{amssymb}
\usepackage[algoruled]{algorithm2e}
\usepackage{tabularx}

%Remove the box from the bottom left from the template
\makeatletter
\def\@copyrightspace{\relax}
\makeatother

\begin{document}

\title{Choice Coordination and Byzantine Agreement}
\subtitle{Feedback Report Randomized Algorithms IN4337}
\numberofauthors{2} 
\author{
\alignauthor
Koos van der Linden \\ \affaddr{4133145}
% 2nd. author
\alignauthor
Marieke van der Tuin \\ \affaddr{4079299}
}
\maketitle
\section{Structure}

The report is readable on its own. The problem is introduced, variables are explained. But not very clearly.

Problem introduction and motivation (motivation missing)
Introduction too short

Pseudocode is useful
Complexity methods are given but not proven.


Experimental Plots are included.



Theoretical analysis before the experimental results

Discussion of the benefit of randomization for the problem (missing)

What is the connection with the analysis and the results

Algorithm 3 does not stop. This is explained later, so refer to it.




\section{Content}


Asynch-CPP is not tested, nor discussed, nor shown that it is equal to Synch-CPP, so we are not convinced. Is it even implemented?

CPP is not discussed theoretically. 

Probability of success of CPP is mentioned but not explained

Choice of $L$, $H$, and $G$ for ByzGen is not explained, nor are these variables explained

Why is t=4/n chosen, with t<n/8 as requirement. n<6 is derived in the analysis. Why is no conclusions drawn from this.
Why do the results show successful instances for t<n/8<n/6<n/4?

Figure 5: these are not averages. Why not?

The number of messages is max $O(n^2)$

References to original publisher of byzgen is missing (only referred to the book).
In this reference it is shown that t<n/4 is the bound. This is a better bound than t=n/6. The conclusion that no better bound can be found is thus incorrect.

Results are not compared with asymptotic bounds, and not with each other.
Results are not checked for validity. Probability of success is not discussed.
No plot is included with n on the horizontal axe to convince that the running time is constant for CCP. Why $n=m=2$? This is not explained

\end{document}
